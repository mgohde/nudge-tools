 \documentclass[12pt,letterpaper]{article}
 
 % Insert package usage statements here:
 \usepackage{listings}
 
 \begin{document}
 
 \title{The Nudge Support Toolset}
 \author{Michael A. Gohde}
 \date{December 6, 2016}
 \maketitle
 
 \section{Overview}
 The task of writing stories for Nudge is inherently difficult. The Nudge
 Support Toolset exists to alleviate some of this difficulty by providing
 a large set of utilities designed to compile, validate, and install stories.
 
 This document exists to detail the function and operation of each utility
 in the Nudge Support Toolset.
 
 \section{story2xml.py}
 story2xml exists to translate well structured, human-readable stories into 
 an intermediate XML format for use by all of the other tools in the set.
 
 \subsection{Input file format}
 \lstset{numbers=left, frame=shadowbox}
 \begin{lstlisting}
 Title: <Insert story title here>
 
 First story node title:
    <Insert story node text here>
    
    Responses:
        Response 1 text -> prb1% to dst1, prbN% to dstN
        Response N text -> prb3% to dst3
 \end{lstlisting}
 
 Each story2xml input file must follow a common set of conventions:
 \begin{enumerate}
   \item The story's title must be written on its own line and prefixed with ``Title:''
   \item Each story ``block'' or ``decision'' (hereafter referred to as a ``node'') must start with a title followed by a colon.
   \item The text content in each node (ie. the story text) must not start with a word followed by a colon.
   \item All possible user actions are defined in a block prefixed by the keyword ``Responses:''. This block must be indented more than all of the other text in its node.
 \end{enumerate}
 
 \subsection{A complete input file example}
 \begin{lstlisting}
Title: Example story

D1_0:
    You are confronted with a serious question: 
    To cheese it or not to cheese it?
    
    Responses:
        Cheese it! -> 50% to D2_0, 50% to D2_1
        Don't cheese it! -> 100% to D2_2

D2_0:
    Note: This is a comment.
    Note: All comment text is discarded.
    
    You were able to cheese it!
    
    Responses:
        Proceed -> 100% to D3_0

D2_1:
    Comment: This is also a comment.
    You were unsuccessful at cheesing it!
    
    Responses:
        Proceed -> 100% to D3_0

D2_2:
    You proceed not to cheese it.
    
    Responses:
        Proceed -> 100% to D3_0
        
D3_0:
    Regardless of whether you cheesed it, 
    something happened.
    
    Responses:
        End -> 100% to END
 \end{lstlisting}
 
 \subsection{Running story2xml}
 story2xml accepts one argument on the command line: the name of a text file containing a properly formatted story.
 Once run, story2xml will print an XML-formatted story to stdout, so it may be useful to redirect its output to a different
 file. 
 
 Example usage:
 \begin{center}
 story2xml.py mystory.txt $>$mystory.xml
 \end{center}
 
 \subsection{Error messages}
 When compiling a story, story2xml attempts to perform a few tests to ensure that the provided source file is logically sound.
 If an error is encountered, story2xml will print a message and information that can be used to locate and correct the error.
 
 List of error messages:
 \paragraph{Error: for response (response text) on line (line number), destination probabilities exceed 100\%}
 This error indicates that the total of all of the weights for some response exceeds 100\%. To correct this error, go to the line indicated and re-evaluate all weights.
 \paragraph{Error: for response (response text) on line (line number), destination probabilities sum to a value below 100\%}
 This error indicates that the total of all o fthe weights for some response is below 100\%. To correct this error, go to the line indicated and re-evaluate all weights.
 \paragraph{Warning: for line (line number), unknown token (string)}
 This message indiciates that the compiler has found a command (a word followed by a colon) that it has no rule to handle. This can safely be ignored in some cases.
 
 \section{sanitytext.py}
 sanitytest exists to check whether a story is logically complete by a number of metrics. These include whether a story can be finished through every possible
 combination of user actions, whether every set of weights sums to 100\%, and whether there are any circular references within a story. 
 
 \subsection{Running sanitytest}
 Like story2xml, sanitytest accepts one argument on the command line: the nane of a text file containing an XML-formatted story file.
 When run, sanitytest will either produce an error output or a message indicating that the story passed all checks.
 \end{document}