\documentclass[12pt,letterpaper]{article}

% Some packages:
\usepackage{listings}
\usepackage{graphicx}
\usepackage{color}

\begin{document}

\title{The Story Editor}
\author{Michael A. Gohde}
\date{April 6, 2017}
\maketitle

\section{Overview}
In order to make story authorship more accessible to a wider range of users, we have developed a graphical story editing and validation tool.
This tool should make it fairly straightforward to edit stories, convert between formats, run validation tasks, and submit drafts for publication.

This document will detail the Story Editor, its features, and its usage in a production environment.

\section{The Editor Window}

When it is first started, the Story Editor will create a blank story and display a window similar to what is depicted above. 
This window has a series of controls intended to allow a user to quickly engage in common tasks, such as adding new story nodes, deleting story nodes, and updating the contents of a given story node.

\begin{figure}
    \begin{center}
        \includegraphics[scale=1]{emptywindow_with_labels.png}
    \end{center}
    \caption{Elements of the editor window}
\end{figure}

\begin{figure}
    \begin{center}
        \includegraphics[scale=1]{examplewindow.png}
    \end{center}
    \caption{Editor window with active story}
\end{figure}

As shown above, the story editor window consists of the following major components:

\paragraph{Story text area}
The story text area displays the formatted text content within the selected story node. This field also allows the user to directly edit the ``source code'' of a story node.

\paragraph{Node List}
This pane contains a list of all nodes currently defined in the story. Each story node is effectively an individual story unit or decision point that is presented to the user.
The node list automatically updates when nodes are added to or removed from the story. 

\paragraph{Graph Area}
The Story Editor is capable of creating a graphical representation of the currently selected node and its relationships within the story.
When the Graph Area is filled, the set of all parent nodes is displayed on the right, the set of all child nodes is displayed on the left, and colored lines are drawn from the currently selected node to its parents and children.

\begin{center}
    \begin{tabular}{l|p{4in}}
    Color & Meaning\\ \hline \hline
    \textcolor{red}{Red} & The connected node does not exist.\\ \hline
    \textcolor{blue}{Blue} & The connected node is an endpoint for the story.\\ \hline
    Black & The connected node exists and is not an endpoint.\\ \hline
    \end{tabular}
    
    
    \textit{List of line colors and their meanings.}
\end{center}

\section{Menu Options}
Much of the validation, import, export, and settings functionality in the Story Editor is contained in its four menus.

\paragraph{The File Menu}
The File menu contains a set of items related to creating, importing, and exporting story lines.

\begin{center}
    \begin{tabular}{l|p{4in}}
    Item & Function\\ \hline \hline
    New & Creates a new story line.\\ \hline
    Open... & Loads a file in the story definition format.\\ \hline
    Save... & Saves a file in the story definition format.\\ \hline
    Import from XML... & Loads a file in the XML story format used by the Nudge Support Tool set.\\ \hline
    Export from XML... & Saves a file in the XML story format.\\ \hline
    Import from Server... & Imports a story line from a Nudge database server.\\ \hline
    Export to Server... & Exports a story line to a set of temporary tables in a Nudge database server.\\ \hline
    Exit & Terminates the application.\\ \hline
    \end{tabular}
    \textit{List of items in the File menu and their functions.}
\end{center}

\paragraph{The Edit Menu}
The Edit menu contains a set of items related to manipulating text and changing various attributes of the current story.

\begin{center}
    \begin{tabular}{l|p{4in}}
    Item & Function\\ \hline \hline
    Title... & Allows the user to specify the current storyline's name.\\ \hline
    Copy & Copies all currently selected text in the current window to the system clipboard.\\ \hline.
    Paste & Inserts all text from the system clipboard into the current editing position.\\ \hline
    \end{tabular}
\end{center}

\paragraph{The Tools Menu}
The Tools menu contains a set of items related to changing global application settings, validating the current story line, and collecting useful data.

\begin{center}
    \begin{tabular}{l|p{4in}}
    Item & Function\\ \hline \hline
    Sanity test & Runs a series of tests on the current story line to determine if it is complete and sufficiently free of errors.\\ \hline
    Publish... & Exports the story to be installed in a production Nudge server.\\ \hline
    Set defaults... & Allows the user to specify various default values for save location, database server credentials, etc.\\ \hline
    Create account on database server... & Creates a user account so that stories can be imported from or saved to a nudge server for easy editing elsewhere.\\ \hline
    \end{tabular}
\end{center}

\section{Editing a story}
The story editing process is fairly straightforward. 

\section{Supported Story Formats}
In order to maintain wide compatibility with many external tool sets and applications, the Story Editor is capable of reading and writing stories in a large number of formats.
Among them, the story definition file, XML, and SQL formats provide the most utility with Nudge tools as implemented.

This section will provide examples of the same story exported to various formats.

\lstset{numbers=left, frame=shadowbox}
\begin{lstlisting}[breaklines=true, caption=Example story in story format.]
Title: Example Story
start:
        This is the first node in the story
        Responses:
                A -> 100% to A
                B -> 50% to B, 50% to C

A:
        You selected option choice A!
        Responses:
                Proceed -> 100% to end

B:
        You chose option B and were taken to node B.
        Responses:
                Proceed -> 100% to end

C:
        You chose option B but were taken to node C!
        Responses:
                Proceed -> 100% to end

end:
        You have reached the end of the story.
        Responses:
                end -> 100% to END
\end{lstlisting}

\begin{lstlisting}[breaklines=true, caption=Example story in XML format.]
<story title="Example Story">
        <node id="start">
                <text>This is the first node in the story</text>
                <answers>
                        <option>
                                <text>A</text>
                                <dest p="100">A</dest>
                        </option>
                        <option>
                                <text>B</text>
                                <dest p="50">B</dest>
                                <dest p="50">C</dest>
                        </option>
                </answers>
        </node>
        <node id="A">
                <text>You selected option choice A!</text>
                <answers>
                        <option>
                                <text>Proceed</text>
                                <dest p="100">end</dest>
                        </option>
                </answers>
        </node>
        <node id="B">
                <text>You chose option B and were taken to node B.</text>
                <answers>
                        <option>
                                <text>Proceed</text>
                                <dest p="100">end</dest>
                        </option>
                </answers>
        </node>
        <node id="C">
                <text>You chose option B but were taken to node C!</text>
                <answers>
                        <option>
                                <text>Proceed</text>
                                <dest p="100">end</dest>
                        </option>
                </answers>
        </node>
        <node id="end">
                <text>You have reached the end of the story.</text>
                <answers>
                        <option>
                                <text>end</text>
                                <dest p="100">END</dest>
                        </option>
                </answers>
        </node>
</story>
\end{lstlisting}

\begin{lstlisting}[breaklines=true, caption=Set of generated SQL statements.]
INSERT INTO tmpstorytable VALUES (1,'Example Story','start','This is the first node in the story',0);
INSERT INTO tmpanswers VALUES ('Example Story','start','A','A');
INSERT INTO tmpresults VALUES (1,'Example Story','start','A',0,100,'A');
INSERT INTO tmpanswers VALUES ('Example Story','start','B','B');
INSERT INTO tmpresults VALUES (2,'Example Story','start','B',0,50,'B');
INSERT INTO tmpresults VALUES (3,'Example Story','start','B',50,100,'C');
INSERT INTO tmpstorytable VALUES (2,'Example Story','A','You selected option choice A!',2);
INSERT INTO tmpanswers VALUES ('Example Story','A','A','Proceed');
INSERT INTO tmpresults VALUES (4,'Example Story','A','A',0,100,'end');
INSERT INTO tmpstorytable VALUES (3,'Example Story','B','You chose option B and were taken to node B.',2);
INSERT INTO tmpanswers VALUES ('Example Story','B','A','Proceed');
INSERT INTO tmpresults VALUES (5,'Example Story','B','A',0,100,'end');
INSERT INTO tmpstorytable VALUES (4,'Example Story','C','You chose option B but were taken to node C!',2);
INSERT INTO tmpanswers VALUES ('Example Story','C','A','Proceed');
INSERT INTO tmpresults VALUES (6,'Example Story','C','A',0,100,'end');
INSERT INTO tmpstorytable VALUES (5,'Example Story','end','You have reached the end of the story.',2);
INSERT INTO tmpanswers VALUES ('Example Story','end','A','end');
INSERT INTO tmpresults VALUES (7,'Example Story','end','A',0,100,'END');
\end{lstlisting}

\section{Remote story editing and storage}

One important possibility that the story editor enables is that of storing story lines on 
the a remote server so that can be more readily merged into Nudge's database. This approach
should also enable users to more easily manage and collaborate on their story lines from multiple
machines.

\subsection{Registering for a Collaborator ID}
In order to import or export story lines to a server, it is necessary to obtain two sets of credentials:
    
\begin{enumerate}
\item A database server login
\item A collaborator login
\end{enumerate}

The database server login allows a user to read and write from the Nudge database itself, while the collaborator login identifies story line ownership and other key information.

In order to obtain a collaborator login, it is necessary to register with the database server. Registration is done
through the ``Create account on database server...'' item in the ``Tools'' menu. 

To register:
    
\begin{enumerate}
\item Open the ``Create account on database server'' dialog
\item Ensure that the database login credentials presented are correct.
\item Enter a desired set of collaborator credentials.
\item Click the ``Check Availability...'' button.
\item If the ``Check Availability...'' button now reads, ``Register...'', then click it and complete the process. Otherwise, change the desired username until it can be registered.
\end{enumerate}

\begin{figure}
    \begin{center}
        \includegraphics[scale=1]{registrationwindow.png}
    \end{center}
    \caption{Collaborator registration box}
\end{figure}

\begin{figure}
    \begin{center}
        \includegraphics[scale=1]{importwindow.png}
    \end{center}
    \caption{Story line import box}
\end{figure}

Once registration is complete, all relevant user authentication information will be stored automatically in the Story Editor's
configuration. This means that all database interactions will automatically use the new credentials, and that the same 
set will be loaded on every start-up.\footnotemark

\footnotetext{It is possible to override this behavior either by using the ``Set Defaults...'' option in the ``Tools'' menu or by 
              creating a new set of login credentials through the ``Create account on database server...'' item.}

\subsection{Storing the current storyline remotely}
With proper credentials, it is possible to save any active storyline to the remote server on which those credentials
were created. This approach to remote storage enables a large number of new possibilities for story editing and collaboration.
It can also be used to safeguard each storyline against data loss, since remote storage enables further points of redundancy.

In order to store a storyline, it is necessary to go through the following steps:
\begin{enumerate}
\item Click the ``Export to Server...'' item in the ``File'' menu. 
\item Ensure that all information presented is correct.
\item If all information is correct, click the ``Export'' button.
\item A copy of the current storyline should now be stored on the remote server.
\end{enumerate}

\subsection{Retrieving a storyline remotely}
The process of importing a storyline is very similar to that of exporting one. 

\section{Publishing a storyline}
Once a storyline is sufficiently edited and checked, it is useful to publish it to the live Nudge server so that it can be
read and interacted with by users. The publication process is fairly straightforward and requires very little additional information
to be entered into the story editor's settings database. 

This is the story publication process:
\begin{enumerate}
\item Click the ``Publish'' option in the ``Tools'' menu.
\item Ensure that the email fields present are correct.
\item If your storyline has passed automatic testing, then it should be possible to able to click the ``Publish'' button.
\item After ``Publish'' is clicked, the system mail client will start with a publication letter filled out.
\item Comment on and revise the email message as appropriate.
\item Send the message.
\end{enumerate}

If all goes well, the storyline should be mailed to an authority capable of checking it over and installing it on a
production Nudge server. 

\begin{figure}
    \begin{center}
        \includegraphics[scale=1]{publicationbox.png}
    \end{center}
    \caption{Story publication dialog box with valid storyline}
\end{figure}

\begin{figure}
    \begin{center}
        \includegraphics[scale=1]{publicationemail.png}
    \end{center}
    \caption{Unedited story email generated by the story editor}
\end{figure}

\section{Story merge support}

While not yet implemented, one important future direction for development is that of enabling
users to merge story lines together. While seemingly limited in usefulness, story line merger
support will enable multiple users to collaboratively edit a scoreline. This, if used correctly, has
the potential to significantly reduce the time needed to write a story line.

\end{document}